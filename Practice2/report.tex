\documentclass[10pt, conference]{IEEEtran}
\usepackage{graphicx}  
\usepackage{amsmath}   
\usepackage{amssymb}   
\usepackage{cite}   
\usepackage{float}    
\usepackage{hyperref}  
\usepackage{booktabs}  
\usepackage{titlesec}  
\usepackage[tableposition=top]{caption}
\usepackage{placeins}

\title{Fetal Head Circumference Semantic Segmentation Report}
\author{Trinh Van Quyet\\ Data Science\\ \texttt{quyetTV.22BI13387@usth.edu.vn}}

\begin{document}
	\maketitle
	
	\begin{abstract}
		The fetal head circumference (HC) is a useful tool to monitor gestational age in fetuses. I implemented a Deeplabv3 pretrained model for computing the HC from 2D ultrasound images using annotation as patterns for model learning to perform a semantic segmentation task.
	\end{abstract}
	
	\titleformat{\section}{\centering\large}{\Roman{section}.}{0.8em}{}
	\titleformat{\subsection}{\normalsize}{\Alph{subsection}.}{0.8em}{}
	
	\section{Introduction}
	To monitor gestational age and fetal growth, ultrasound imaging is used to measure fetal head circumference (HC). The HC18 challenge provides a dataset for automating HC computation from ultrasound images. I decide to approach the HC18 challenge and build a DeepLabv3 pretrained model to segment the fetal head masks and then compute to get results.
	
	\section{Methodology}
	\subsection{Data Exploration}
	The dataset have 999 ultrasound images and their correspond annotations in training set. On the other hand, there are 335 images in test set.
	
	The first, I visualize both ultrasound images and their correspond annotation images on a figure to whether their annotation is right.
	
	\begin{figure}[H]
		\centering
		\includegraphics[width=0.5\textwidth]{D:/USTH_SUBJECTS/B3/MachineLearningInMedicine/mlmed2025/Practice2/figures/EDA.png}
		\caption{Visualize Ultrasound Images and Their Annotations}
		\label{fig:visualize}
	\end{figure}
	
	The next, I conduct to extracting the properties of image such as filename, height, width, channel and format.
	
	\begin{table}[h]
		\centering
		\begin{tabular}{|c|c|c|c|c|}
			\hline
			\textbf{Filename} & \textbf{Height} & \textbf{Width} & \textbf{Channel} & \textbf{Format} \\ \hline
			000\_HC.png & 540 & 800 & 3 & png \\ \hline
			001\_HC.png & 540 & 800 & 3 & png \\ \hline
			002\_HC.png & 540 & 800 & 3 & png \\ \hline
			003\_HC.png & 540 & 800 & 3 & png \\ \hline
			004\_HC.png & 540 & 800 & 3 & png \\ \hline
			005\_HC.png & 540 & 800 & 3 & png \\ \hline
			
		\end{tabular}
		\caption{Image Properties}
		\label{tab:properties}
	\end{table}
	
	\subsection{Data Pre-processing}
\end{document}